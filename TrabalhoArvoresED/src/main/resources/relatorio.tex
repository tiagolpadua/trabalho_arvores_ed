\documentclass[a4paper,11pt]{article}

%Packages
\usepackage[brazilian]{babel} %Brazilian Portuguese
\usepackage[T1]{fontenc}
\usepackage{subfig}
\usepackage{color}
\usepackage[utf8]{inputenc}
\usepackage{url} %URLs
\usepackage{makeidx} %Index
\usepackage[pdftex]{graphicx} %Graphics
\usepackage{amsfonts} %Math fonts
%\usepackage{indentfirst} %Makes it indent the first paragraph of the section
\usepackage{listings} %Code support, properly highlighted
\usepackage{verbatim} %Better verbatim
%\usepackage{fancyvrb} %Fancy verbatim
\usepackage{cite} %Better citation
%\usepackage{siunitx} %SI Units 
\usepackage{hyperref} %Makes links to your references, making your life quite a bit easier.
%\usepackage[pdftex]{colortbl} %Color Tables
\usepackage{array} %Better tables - improves the algorythms
\hypersetup{ %Sets up hyperref
    %bookmarks=true,         % show bookmarks bar?
    %unicode=false,          % non-Latin characters in Acrobat?s bookmarks
    %pdftoolbar=true,        % show Acrobat?s toolbar?
    %pdfmenubar=true,        % show Acrobat?s menu?
    %pdffitwindow=false,     % window fit to page when opened
    %pdfstartview={FitH},    % fits the width of the page to the window
    pdftitle={Relatório},    % title
    %pdfauthor={Felipe Brandão Cavalcanti},     % author
    colorlinks=false,       % false: boxed links; true: colored links
    linkcolor=red,          % color of internal links
    citecolor=green,        % color of links to bibliography
    filecolor=magenta,      % color of file links
    urlcolor=cyan           % color of external links
}
\lstset{ %Sets up lisitings, so we get highlighted code
language=Java,                     % choose the language of the code
basicstyle=\footnotesize,       % the size of the fonts that are used for the code
numbers=left,                   % where to put the line-numbers
numberstyle=\footnotesize,      % the size of the fonts that are used for the line-numbers
stepnumber=2,                   % the step between two line-numbers. If it's 1 each line will be numbered
numbersep=5pt,                  % how far the line-numbers are from the code
backgroundcolor=\color{white},  % choose the background color. You must add \usepackage{color}
showspaces=false,               % show spaces adding particular underscores
showstringspaces=false,         % underline spaces within strings
showtabs=false,                 % show tabs within strings adding particular underscores
frame=single,	                % adds a frame around the code
tabsize=2,	                    % sets default tabsize to 2 spaces
captionpos=b,                   % sets the caption-position to bottom
breaklines=true,                % sets automatic line breaking
breakatwhitespace=false,        % sets if automatic breaks should only happen at whitespace
escapeinside={\%*}{*)}          % if you want to add a comment within your code
}


\parindent15pt  \parskip0pt
\setlength\voffset{-2.0cm}
\setlength\hoffset{-1.5cm}
\setlength\textwidth{16.0cm}
\setlength\textheight{24.5cm}
\setlength\baselineskip{2cm}
\renewcommand{\baselinestretch}{1.2}

\newcommand{\HRule}{\rule{\linewidth}{0.5mm}}

\begin{document}
\begin{titlepage}
\begin{center}
 
% Upper part of the page
\includegraphics[width=0.25\textwidth]{./unb.pdf}\\[1cm]
 
\textsc{\LARGE Universidade de Brasília}\\[1.5cm]
 
\textsc{\Large Estrutura de Dados - Turma B - Trabalho I}\\[0.5cm]
 
% Title
\HRule \\[0.4cm]
{ \huge \bfseries Árvores de Jogos\\$Game Trees$}
\HRule
\vspace{0.75cm}
\large CIC - Departamento de Ciência da Computação\\
\vspace{0.8cm}
% Author and supervisor
\begin{minipage}{0.4\textwidth}
\begin{flushleft} \large
\emph{Autores:}\\
Tiago L. P. de Pádua - 12/1042457\\
Ronaldo S. Ferreira Jr. - 09/48721\\
Alex Leite - 05/97694\\
\end{flushleft}
\end{minipage}
\begin{minipage}{0.4\textwidth}
\begin{flushright} \large
\emph{Professor:} \\
Eduardo A. P. Alchieri
\end{flushright}
\end{minipage}
 
\vfill
 
% Bottom of the page
{\large Fevereiro de 2013}
\end{center}
\end{titlepage}

\pagestyle{plain}

\section{Introdução} 
Através do estudo de estrututa de dados, este trabalho tem o objetivo a confecção de um projeto, onde deverá ser feito um $software$ que utilize uma  árvore genérica, chamada de $game tree$, para implementar a inteligência artificial de um jogo de tabuleiro utilizando o algoritmo conhecido como $Minimax$.

Neste projeto foi utilizado o jogo de tabuleiro Mancala (\hyperlink{http://pt.wikipedia.org/wiki/Mancala}{ http://pt.wikipedia.org/wiki/Mancala}), onde o objetivo geral do jogo é mover o maior número de peças para a "vala" do jogador, que é a cavidade mais a direita do tabuleiro (demais regras do jogo podem ser obtidas no link citado).

O algoritmo $Minimax$ é utilizado para indicar qual a melhor jogada a ser realizada a partir da disposição atual do tabuleiro utilizando uma árvore de jogadas possíveis. A idéia central do algoritmo é que os jogadores sempre escolherão a jogada que lhe traga maior benefício em detrimento do oponente. (\hyperlink{http://en.wikipedia.org/wiki/Minimax}{http://en.wikipedia.org/wiki/Minimax}

\section{Implementação}
A implementação do código foi realizada utilizando a linguagem de programação Java (javac 1.7.0\_13) através da plataforma de desenvolvimento Netbeans 7.2.1.\\

As seguintes classes foram criadas:
\begin{itemize}
  \item Arvore.java   
  \item ListaLigada.java
  \item MancalaConsole.java
  \item No.java
  \item Elemento.java
  \item Mancala.java
  \item MancalaException.java
\end{itemize}

\subsection{MancalaConsole.java}
Classe que possui a função $main$ do programa, ou seja, a primeira função a ser chamada quando o programa é executado, sua função principal é fornecer uma interface via linha de comando pela qual o usuário irá interagir com o sistema;

\subsection{MancalaException.java}
Utilizada para indicar que ocorreu um erro dentro da execução do programa.

\subsection{Arvore.java}
Classe que implementa as funções básicas de uma árvore, possui somente o campo $raiz$ do tipo $No$ e uma função que retorna um booleano caso a árvore esteja vazia.

\subsection{No.java}
Representa um nó de uma árvore, possui um valor (utilizado pela função $Minimax$), um booleano $alpha$ que indica de qual jogador é a "vez" de jogar, um objeto do tipo No chamado de "pai", que é uma referência para o nó "pai", necessário uma vez que o algoritmo $Minimax$ é executado no sentido das folhas para a raiz da árvore, além disso, possui também uma lista ligada que representa os "filhos" deste nó e um número inteiro que representa qual casa foi movida na mancala para que ela chegasse no estado atual.

\subsection{ListaLigada.java}
Implementa uma lista encadeada que possui uma referência para o primeiro elemento da lista além das funcões utilitárias de inserção na lista e verificação de lista vazia.

\subsection{Elemento.java}
Classe utilizada como unidade básica da lista ligada.

\subsection{Mancala.java}
Representa o tabuleiro do jogo, contém a "inteligência" necessária para mover as peças de uma casa e redistribuí-las corretamente conforme as regras do jogo e, após a movimentação das peças, indica de quem é a "vez de jogar", indica também se o jogo terminou. Além disso, nesta classe também está implementado o algoritmo $Minimax$, ressaltamos que a questão mais importante do algoritmo é a função de avaliação da Mancala. Em nosso projeto, definimos que a avaliação da Mancala é realizada pela diferença entre a quantidade de peças dos jogadores em suas "valas", assim, o computador irá escolher jogadas que maximizem o número de peças em sua "vala" e minimize do oponente.

\subsection{Execução do programa}
Quando o programa é executado, inicialmente ele solicita que o usuário informe qual o número de níveis devem ser utilizados na criação da árvore de jogadas, em nossos testes, valores acima de 8 tornaram o sistema instável devido à utilização de toda a memória do computador.

É então instanciado um novo objeto do tipo Mancala, que representa o tabuleiro de jogadas, então inicia-se um loop que é executado enquanto o jogo não terminar (esta informação é obtida diretamente do objeto Mancala).

O próximo passo do programa é identificar de quem é a "vez de jogar", caso seja do jogador humano, é solicitado que este informe a casa a ser movida, caso seja a vez do computador, são seguidos os seguintes passos:

\begin{enumerate}
\item Cria-se uma árvore de jogadas a partir do estado atual da Mancala com o número de níveis informados pelo usuário;
\item Aplica-se o algoritmo $Minimax$ à árvore gerada;
\item Identifica-se na árvore resultante qual deve ser o movimento adotado pelo computador;
\item O computador move a casa indicada pelo algoritmo;
\item São impressos em tela o tempo gasto de processamento e o número de jogadas analisadas;
\end{enumerate}

\subsection{Testes}
Foram criadas as classes de teste "ArvoreTeste.java" e "MinimaxTeste.java", nestes testes foi utilizado como referência o exemplo da aplicação do algoritmo $Minimax$ cuja imagem de nome "701px-Minimax.svg.png" se encontra junto do projeto, foi comparada a árvore resultante e identificamos que estas eram idênticas.

\section{Conclusão}
Pode-se concluir que a utilização árvores de jogadas em conjunto com o algoritmo $Minimax$ é uma boa alternativa para se implementar inteligência artificial em jogos de tabuleiro, no entanto, o fator limitante é o número de jogadas disponíveis para os jogadores, uma vez que em jogos mais complexos como o Xadrez, a árvore de jogadas se tornará muito grande em um pequeno número de níveis, devendo assim ser utilizado um algoritmo de "poda" para economizar a memória do sistema.

\nocite{*}					%Imprime toda a bibliografia
\bibliography{bibliografia}		
\bibliographystyle{plain}		
\end{document}
