\documentclass[a4paper,11pt]{article}

%Packages
\usepackage[brazilian]{babel} %Brazilian Portuguese
\usepackage[T1]{fontenc}
\usepackage{subfig}
\usepackage{color}
\usepackage[utf8]{inputenc}
\usepackage{url} %URLs
\usepackage{makeidx} %Index
\usepackage[pdftex]{graphicx} %Graphics
\usepackage{amsfonts} %Math fonts
%\usepackage{indentfirst} %Makes it indent the first paragraph of the section
\usepackage{listings} %Code support, properly highlighted
\usepackage{verbatim} %Better verbatim
%\usepackage{fancyvrb} %Fancy verbatim
\usepackage{cite} %Better citation
%\usepackage{siunitx} %SI Units 
\usepackage{hyperref} %Makes links to your references, making your life quite a bit easier.
%\usepackage[pdftex]{colortbl} %Color Tables
\usepackage{array} %Better tables - improves the algorythms
\hypersetup{ %Sets up hyperref
    %bookmarks=true,         % show bookmarks bar?
    %unicode=false,          % non-Latin characters in Acrobat?s bookmarks
    %pdftoolbar=true,        % show Acrobat?s toolbar?
    %pdfmenubar=true,        % show Acrobat?s menu?
    %pdffitwindow=false,     % window fit to page when opened
    %pdfstartview={FitH},    % fits the width of the page to the window
    pdftitle={Relatório},    % title
    %pdfauthor={Felipe Brandão Cavalcanti},     % author
    colorlinks=false,       % false: boxed links; true: colored links
    linkcolor=red,          % color of internal links
    citecolor=green,        % color of links to bibliography
    filecolor=magenta,      % color of file links
    urlcolor=cyan           % color of external links
}
\lstset{ %Sets up lisitings, so we get highlighted code
language=Java,                     % choose the language of the code
basicstyle=\footnotesize,       % the size of the fonts that are used for the code
numbers=left,                   % where to put the line-numbers
numberstyle=\footnotesize,      % the size of the fonts that are used for the line-numbers
stepnumber=2,                   % the step between two line-numbers. If it's 1 each line will be numbered
numbersep=5pt,                  % how far the line-numbers are from the code
backgroundcolor=\color{white},  % choose the background color. You must add \usepackage{color}
showspaces=false,               % show spaces adding particular underscores
showstringspaces=false,         % underline spaces within strings
showtabs=false,                 % show tabs within strings adding particular underscores
frame=single,	                % adds a frame around the code
tabsize=2,	                    % sets default tabsize to 2 spaces
captionpos=b,                   % sets the caption-position to bottom
breaklines=true,                % sets automatic line breaking
breakatwhitespace=false,        % sets if automatic breaks should only happen at whitespace
escapeinside={\%*}{*)}          % if you want to add a comment within your code
}


\parindent15pt  \parskip0pt
\setlength\voffset{-2.0cm}
\setlength\hoffset{-1.5cm}
\setlength\textwidth{16.0cm}
\setlength\textheight{24.5cm}
\setlength\baselineskip{2cm}
\renewcommand{\baselinestretch}{1.2}

\newcommand{\HRule}{\rule{\linewidth}{0.5mm}}

\begin{document}
\begin{titlepage}
\begin{center}
 
% Upper part of the page
\includegraphics[width=0.25\textwidth]{./unb.pdf}\\[1cm]
 
\textsc{\LARGE Universidade de Brasília}\\[1.5cm]
 
\textsc{\Large Estrutura de Dados - Turma B - Trabalho II}\\[0.5cm]
 
% Title
\HRule \\[0.4cm]
{ \huge \bfseries Game Trees}
\HRule
\vspace{0.75cm}
\large CIC - Departamento de Ciência da Computação\\
\vspace{0.8cm}
% Author and supervisor
\begin{minipage}{0.4\textwidth}
\begin{flushleft} \large
\emph{Autores:}\\
Tiago L. P. de Pádua - 12/1042457\\
Ronaldo S. Ferreira Jr. - 09/48721\\
Alex Leite - 05/97694\\
\end{flushleft}
\end{minipage}
\begin{minipage}{0.4\textwidth}
\begin{flushright} \large
\emph{Professor:} \\
Eduardo A. P. Alchieri
\end{flushright}
\end{minipage}
 
\vfill
 
% Bottom of the page
{\large Fevereiro de 2013}
\end{center}
\end{titlepage}

\pagestyle{plain}

%\begin{abstract}
%Através do estudo de estrututa de dados, este trabalho tem o objetivo a confecção de um projeto, onde deverá ser feito um $software$ capaz de de validar, avaliar e realizar a conversão de uma expressão aritmética escrita em forma infixa para a forma posfixa.
%\end{abstract}

\section{Introdução} 
Por meio dos estudos e utilização de árvores genéricas, será implementado um jogo de tabuleiro, com inteligencia artificial, em que o computador será capaz de prever os movientos do jogador humano.

O jogo a ser implementado é de origem africana, denominado $Mancala$, embora existam diversas modalidades e regras, todas elas seguem o padrão em que o tabuleiro possui 14 casas, as quais 12 são casas menores, divididas em 2 fileiras de 6, sendo que cada fileira pertence a um jogadoe e, 2 casas maiores, chamadas $Khalas$. [INSERIR FIGURA, INSERIR RESTANTE DAS REGRAS]

[EXPLICAR GAME TREE]
[EXPLICAR MINIMAX] 



\section{Implementação}
A implementação do código foi realizada utilizando a linguagem de programação Java (javac 1.6.0\_37) através da plataforma de desenvolvimento Netbeans 7.0.1.\\

As seguintes classes foram criadas:
\begin{itemize}
  \item CLASSES
\end{itemize}

\section{CÓDIGO FONTE}
Para a realização dos testes, foir criada a classe TesteExpressaoAritmetica.java e foram realizados os testes conforme consta abaixo:
\begin{lstlisting}
[CODIGO FONTE]
\end{lstlisting}

\section{Conclusão}
Conclui-se que a utilização de árvores genéricas é de grande ajuda para desenvolvimento de sistemas com inteligencia atificial, predição de resultados e movimentos, tais como jogos.
(EM PROCESSO DE CONCLUSÃO)


\nocite{*}					%Imprime toda a bibliografia
\bibliography{bibliografia}		
\bibliographystyle{plain}		
\end{document}